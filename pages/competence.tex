\section*{Kompetenzbereiche}

Im Rahmen des Projekts werden die farbig hervorgehobenen Kompetenzbereiche gezielt vermittelt. Diese Inhalte sind fester Bestandteil des Lehrplans und stellen sicher, dass alle Lernenden die grundlegenden Kenntnisse und Fertigkeiten in den Bereichen Elektronik, Mechanik, Firmware und Prototyping erwerben. Welche Kompezenzbereiche im Projekt vermittelt werden, kann im Detail den nachfolgenden Tabellen entnommen werden.

\vspace{-10pt}

\begin{figure}[ht]
  \centering
  \begin{tikzpicture}[scale=1.25]
    % % Kreisradius
\usetikzlibrary{decorations.text}

\fill[green!20,opacity=0.45]  (0,0) -- (90:\catAvalue+1)  arc (90:180:\catAvalue+1)  -- cycle;
\fill[red!20,opacity=0.45]    (0,0) -- (0:\catBvalue+1)   arc (0:90:\catBvalue+1)    -- cycle;
\fill[yellow!30,opacity=0.45] (0,0) -- (270:\catCvalue+1) arc (270:360:\catCvalue+1) -- cycle;
\fill[blue!20,opacity=0.45]   (0,0) -- (180:\catDvalue+1) arc (180:270:\catDvalue+1) -- cycle;
\fill[white]    (0,0) -- (0:1)   arc (0:360:1)    -- cycle;


\ifnum\catAOptionalvalue>0
  \fill[gray!30,opacity=0.25]
    (90:\catAvalue+1) arc (90:180:\catAvalue+1) --
    (180:\catAvalue+1+\catAOptionalvalue) arc (180:90:\catAvalue+1+\catAOptionalvalue)
    -- cycle;
\fi

\ifnum\catBOptionalvalue>0
  \fill[gray!30,opacity=0.25]
    (0:\catBvalue+1) arc (0:90:\catBvalue+1) --
    (90:\catBvalue+1+\catBOptionalvalue) arc (90:0:\catBvalue+1+\catBOptionalvalue)
    -- cycle;
\fi

\ifnum\catCOptionalvalue>0
  \fill[gray!30,opacity=0.25]
    (270:\catCvalue+1) arc (270:360:\catCvalue+1) --
    (360:\catCvalue+1+\catCOptionalvalue) arc (360:270:\catCvalue+1+\catCOptionalvalue)
    -- cycle;
\fi

\ifnum\catDOptionalvalue>0
  \fill[gray!30,opacity=0.25]
    (180:\catDvalue+1) arc (180:270:\catDvalue+1) --
    (270:\catDvalue+1+\catDOptionalvalue) arc (270:180:\catDvalue+1+\catDOptionalvalue)
    -- cycle;
\fi

\pgfmathsetmacro{\maxval}{\numberOfsubcategories + 1}

% Hilfskreise
\foreach \r in {1,...,\maxval} {
  \draw[gray!30] (0,0) circle (\r);
}

% Achsen und Kategorien
\foreach \i [count=\idx from 1] in \categories {
  \draw[gray!60] (0,0) -- (90-90*\idx:\maxval);

  \pgfmathsetmacro{\startang}{450-180*\idx}
  \pgfmathsetmacro{\rr}{
    ifthenelse(\startang < 0, 5.75, 5.15)
  }
  \path[
    postaction={decorate},
    decoration={
      text along path,
      text={|\Large\sffamily|\i},
      text align={center},
      raise=1.0ex,
    }
  ]
  (\startang:\rr) arc (\startang:0:\rr);

}

\foreach \winkel/\liste in \subcategories{

  \pgfmathsetmacro{\rstart}{
    ifthenelse(\winkel < 0, 0.75, 0.25)
  }

  \foreach \text [count=\i from 1] in \liste {
    \pgfmathsetmacro{\r}{\rstart+\i}
    \path[
      postaction={decorate},
      decoration={
        text along path,
        text={|\footnotesize\sffamily|\text},
        text align={center},
        raise=1.0ex,
      }
    ]
    (\winkel:\r) arc (\winkel:0:\r);
  }
}

% \pgfmathsetmacro{\degree}{-90}
% \foreach \text [count=\i from 1] in {Flashen,Erstellen,Erweitern,Algorithmen} {
%   \pgfmathsetmacro{\r}{0.85 + \i}
%   \path[
%     postaction={decorate},
%     decoration={
%       text along path,
%       text={|\footnotesize\sffamily|\text},
%       text align={center},
%       raise=1.0ex,
%     }
%   ]
%   (\degree:\r) arc (\degree:0:\r);
% }

% \pgfmathsetmacro{\degree}{-270}
% \foreach \text [count=\i from 1] in {Slicintg,3D-Druck,Modellierung,Laser-Cutting} {
%   \pgfmathsetmacro{\r}{0.85 + \i}
%   \path[
%     postaction={decorate},
%     decoration={
%       text along path,
%       text={|\footnotesize\sffamily|\text},
%       text align={center},
%       raise=1.0ex,
%     }
%   ]
%   (\degree:\r) arc (\degree:0:\r);
% }

% \pgfmathsetmacro{\degree}{90}
% \foreach \text [count=\i from 1] in {Vorbearbeiten,Spanende-Bearbeitung,Spanlose-Bearbeitung,CNC} {
%   \pgfmathsetmacro{\r}{0.25 + \i}
%   \path[
%     postaction={decorate},
%     decoration={
%       text along path,
%       text={|\footnotesize\sffamily|\text},
%       text align={center},
%       raise=1.0ex,
%     }
%   ]
%   (\degree:\r) arc (\degree:0:\r);
% }

% \pgfmathsetmacro{\degree}{270}
% \foreach \text [count=\i from 1] in {Transistor,Analog,Digital,Mikrocontroller} {
%   \pgfmathsetmacro{\r}{0.25 + \i}
%   \path[
%     postaction={decorate},
%     decoration={
%       text along path,
%       text={|\footnotesize\sffamily|\text},
%       text align={center},
%       raise=1.0ex,
%     }
%   ]
%   (\degree:\r) arc (\degree:0:\r);
% }

  \end{tikzpicture}
  \caption{Darstellung der Kompetenzbereiche in einem Radardiagramm}
  \label{fig:kompetenzbereiche-radardiagramm}
\end{figure}

Die grau markierten (vertieften) Kompetenzbereiche können im Projekt ebenfalls bearbeitet werden. Das abarbeiten vertiefter Kompetenzbereiche ist freiwillig und kann auch (sofern möglich) außerhalb der regulären Unterrichtszeit stattfinden. Das abarbeiten vertiefter Kompetenzen sichert dem Lernenden zusätzliche Punkte im Feedback der Übung. Die grau markierten Bereiche zum vertieften Kompetenzerwerb sind im Projektverlauf deutlich gekennzeichnet.

\thispagestyle{empty}
\newpage

\renewcommand{\do}[1]{\showCompetenceTable{#1}}
\expandafter\docsvlist\expandafter{\categories}
