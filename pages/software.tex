\newcounter{softwaretnotecounter}

\section*{Benötigte Software}
\begin{table}[!ht]
  \renewcommand{\arraystretch}{1.3}
  \centering
  \begin{threeparttable}
    \begin{tabularx}{\linewidth}{|M{2.4cm}|M{2.2cm}|X|}
      \hline
      \rowcolor{gray!20}
      \textbf{Abbildung\tnote{*}} & \textbf{Name} & \textbf{Beschreibung} \\
      \hline
\ifsoftwarepagelineartechnology
      \stepcounter{softwaretnotecounter}
      \raisebox{-.25\height}{\includesvg[width=1.6cm]{./logo/linear_technology}} &
      LT-Spice\tnote{(\thesoftwaretnotecounter)} &
      Simulationssoftware für elektronische Schaltungen, weitverbreitet zum Entwurf und Testen von Schaltungen.\\
      \hline
\fi
\ifsoftwarepagemicrochipstudio
      \stepcounter{softwaretnotecounter}
      \raisebox{-.25\height}{\includesvg[width=1.6cm]{./logo/microchip}} &
      Microchip Studio\tnote{(\thesoftwaretnotecounter)} &
      Entwicklungsumgebung (IDE) zur Programmierung von Mikrocontrollern, insbesondere AVR und ARM.\\
      \hline
\fi
\ifsoftwarepagefreecad
      \stepcounter{softwaretnotecounter}
      \raisebox{-.25\height}{\includesvg[width=0.65cm]{./logo/freecad}} &
      FreeCAD\tnote{(\thesoftwaretnotecounter)} &
      Open-Source CAD-Software für 3D-Modellierung, geeignet für technische Konstruktionen und Design.\\
      \hline
\fi
\ifsoftwarepagedremeldigilabslicer
      \stepcounter{softwaretnotecounter}
      \raisebox{-.25\height}{\includesvg[width=1.6cm]{./logo/dremel}} &
      Dremel 3D DigiLab Slicer\tnote{(\thesoftwaretnotecounter)} &
      Software zum Vorbereiten und Slicen von 3D-Druck-Modellen, speziell für Dremel 3D-Drucker optimiert.\\
      \hline
\fi
    \end{tabularx}
    \caption{Softwarekomponenten für das Projekt}
    \label{tab:software-programme}
    \begin{tablenotes}
      \footnotesize
      \item[*] Die abgebildeten Logos sind markenrechtlich geschützte Symbole. Sie werden in dieser Dokumentation ausschließlich zur Identifikation der jeweiligen Softwareprodukte verwendet, ohne Werbeabsicht.
      \item[~]
\setcounter{softwaretnotecounter}{0}
\ifsoftwarepagelineartechnology
      \stepcounter{softwaretnotecounter}
      \item[(\thesoftwaretnotecounter)] \url{https://www.analog.com/en/resources/design-tools-and-calculators.html}
\fi
\ifsoftwarepagemicrochipstudio
      \stepcounter{softwaretnotecounter}
      \item[(\thesoftwaretnotecounter)] \url{https://www.microchip.com/en-us/tools-resources/develop/microchip-studio}
\fi
\ifsoftwarepagefreecad
      \stepcounter{softwaretnotecounter}
      \item[(\thesoftwaretnotecounter)] \url{https://www.freecad.org/downloads.php}
\fi
\ifsoftwarepagedremeldigilabslicer
      \stepcounter{softwaretnotecounter}
      \item[(\thesoftwaretnotecounter)] \url{https://www.dremel.com/at/de/digilab/3d-druckersoftware}  
\fi
    \end{tablenotes}
  \end{threeparttable}
\end{table}